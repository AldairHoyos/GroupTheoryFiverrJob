\documentclass[a4paper,openany,11pt]{book}
\usepackage[english]{babel}
\usepackage[utf8]{inputenc}
\usepackage{amsmath}
\usepackage{array,multirow}
\usepackage{graphicx}


\setcounter{secnumdepth}{3} %para que ponga 1.1.1.1 en subsubsecciones
\setcounter{tocdepth}{3} % para que ponga subsubsecciones  en el indice

\usepackage[left=2cm,top=2cm,right=2cm,bottom=2cm]{geometry} %Definimos los Margenes


\usepackage{fancyhdr}
\pagestyle{fancy}

\fancyhead[LO]{\leftmark} % En las páginas impares, parte izquierda del encabezado, aparecerá el nombre de capítulo
\fancyhead[RE]{\rightmark} % En las páginas pares, parte derecha del encabezado, aparecerá el nombre de sección
\renewcommand{\chaptermark}[1]{\markboth{\textbf{\thechapter. #1}}{}} % Formato para el capítulo: N. Nombre
\renewcommand{\sectionmark}[1]{\markright{\textbf{\thesection. #1}}} % Formato para la sección: N.M. Nombre

\usepackage{blindtext}
\usepackage{graphicx}
\usepackage{mathpazo}
\usepackage{tikz}
\usepackage{latexsym,amsmath,amssymb,amsfonts,cancel}


\newcommand{\titulo}[1]{\vspace{4cm}{\Huge\textsc{\begin{center}#1 \end{center}\vspace{-28pt}}}}
\newcommand{\subtitulo}[1]{{\Large\textsc{\begin{center}#1\\[5cm]\end{center}}}}
\newcommand{\tutor}[1]{{\Large\textsc{\begin{center}Tutor: #1\\\end{center}}}}
\newcommand{\lugarfecha}{{\Large\textsc{\begin{center}Valencia-Venezuela\\ \today\end{center}}}\vfill}
\newcommand{\logos}{\begin{tikzpicture}[remember picture,overlay]

% draw image
\node[inner sep=0, xshift=2.9cm, yshift=-3.7cm] at (current page.north west)
{\includegraphics[width=1.5cm]{uc.jpg}};
\node[inner sep=0, xshift=-2.9cm, yshift=-3.7cm] at (current page.north east)
{\includegraphics[width=1.5cm]{facyt.jpg}};
\draw (0,-3.3cm) -- (17cm,-3.3cm) ;
\node[align=center, scale=1.4] at (0.5\textwidth,-1.6cm)
{República Bolivariana de Venezuela\\ Universidad de Carabobo\\ Facultad Experimental de Ciencias y Tecnología\\ Departamento de Matemáticas};
	
%\draw[gray!70!blue] (0,-4.0cm) -- (17cm,-4.0cm) ;
\end{tikzpicture}}



\setlength{\parindent}{0cm} 

\newtheorem{Teo}{Theorem}
\newtheorem{Def}{Definition}
\newtheorem{Lemma}{Lemma}
\newtheorem{Prob}{Problem}
\newtheorem{Corol}{Corollary}


\begin{document}
	
	\begin{center}
		\text{\bf \Large \underline{Group Theory}}
	\end{center}

\begin{Prob}
	Find all cyclic generators for $(U_{37}, \otimes, 1)$.
\end{Prob}

\textbf{Solution:} Let us consider the group $(U_{37}, \otimes, 1)$. The elements of this group are all positive integers less than $37$ that are relative primes with respect to $37$. Now, $37$ is prime so the only divisors of $37$ are $1$ and $37$. Therefore, for all integers $1 \leq x < 37$, we have that $gcd(x,37) = 1$. Thus, we can exhibit all elements of $(U_{37}, \otimes, 1)$ in the following way:

\begin{center}
	$U_{37} = \lbrace n \in \mathbb{Z}^{+} : n \leq 36 \rbrace \implies \left|U_{37}\right| = 36$
\end{center} 

Now, consider the following auxiliary result which will allow us to find all cyclic generators of $U_{37}$:

\begin{Lemma}
	Suppose that $p$ is a prime number and $a \in (U_{37},\otimes, 1)$. Then $a$ is a generator of $(U_{37},\otimes, 1)$ if and only if:
	
	\begin{center}
		$a^{\frac{p-1}{q}} \not \equiv 1 \mod p$
	\end{center}

	for all primes $q$ such that $q \mid (p-1)$.
\end{Lemma}

So, knowing that $36 = 2^{2}\cdot3^{2}$, we know that all the prime numbers that divide $36$ are $2$ and $3$. Therefore, we only need to find those elements $a \in 
U_{37}$ that satisfy the following two conditions simultaneously:

\begin{center}
	$a^{18} \not \equiv 1 \mod 37$ \hspace{0.1cm} $\wedge$ \hspace{0.1cm} $a^{12} \not \equiv 1 \mod 37$
\end{center}

For example $2$ is a generator of $(U_{37},\otimes, 1)$ because we have that:

\begin{center}
	$2^{18} \equiv 36 \mod 37$ \hspace{0.1cm} $\wedge$ \hspace{0.1cm} $a^{12} \equiv 26 \mod 37$
\end{center}

Following a similar approach we have that the following list represent the entire list of elements that generate $(U_{37},\otimes, 1)$, this is:

\begin{center}
	$\lbrace 2, 5, 13, 15, 17, 18, 19, 20, 22, 24 , 32, 35 \rbrace$\hspace{0.1cm} $\square$
\end{center}

\begin{Prob}
	Find all subgroups of size $9$ in $(U_{37,\oplus,1})$.
\end{Prob}

\textbf{Solution:} 

\begin{Prob}
	Let
	
	\begin{center}
		$G = \biggl\{\left[\begin{array}{cc}
		a & 3b\\
		b & a
		\end{array}\right] : a$ and $b$ are real numbers and $a\neq 0$ or $b \neq 0\biggr\} $
	\end{center}

	Prove that $G$ is not a group with respect to multiplication.

\end{Prob}

\textbf{Proof:} In order to show that $G$ is not a group, we need to show that $G$ (together with the matrix multiplication as the set operation) does not satisfy any of the following group conditions:

\begin{enumerate}
	\item \textbf{Closure Law}: If $A$ and $B$ are elements of $G$, then $AB$ is also an element of $G$. 
	
	\item \textbf{Associative Law}: If $A, B$ and $C$ are elements of $G$, then $(AB)C = A(BC)$.
	
	\item \textbf{Existence of unit}: There is an element $I$ in $G$ such that for all $A$ in $G$, we have that $IA = AI = A$.
	
	\item \textbf{Existence of inverse}: For every element $A$ in $G$, there is another element $B$ in $G$ such that $AB = BA = I$.
\end{enumerate} 

If $G$ does not satisfy any of the above conditions, then $G$ is not a group. Let us show that $G$ together with the matrix multiplication as the set operation for $G$ does not satisfy condition $4$.\\

Effectively, let us take a matrix from the set $G$. Then, the form of such matrix is the following:

\begin{center}
	$\left[\begin{array}{cc}
	a & 3b\\
	b & a
	\end{array}\right]$, \hspace{0.1cm} where $a, b \in \mathbb{R}$ and $ab \neq 0$.
\end{center}

Now, the previous matrix has a multiplicative inverse, if and only if, its determinant is different from zero. Let us compute the determinant of such matrix, this is:

\begin{center}
	$\left|\begin{array}{cc}
	a & 3b\\
	b & a
	\end{array}\right| = a^{2} - 3b^{2}$
\end{center}

Therefore, once we have the definitive form of the determinant of any matrix in $G$ (as in the previous paragraph shows), we can now determine the condition that will make such determinant equal to zero. Let us consider the following:

\begin{center}
	$\left|\begin{array}{cc}
	a & 3b\\
	b & a
	\end{array}\right| = a^{2} - 3b^{2} = 0 \iff  a^{2} = 3b^{2} \iff \pm a = \pm\sqrt{3}b$
\end{center} 

Thus, we have shown that if the above matrix is such that $\pm a = \pm\sqrt{3}b$, then its determinant is equal to zero. Now, the following matrix is a member of the set $G$:

\begin{center}
	$A = \left[\begin{array}{cc}
	\sqrt{3}b & 3b\\
	b & \sqrt{3}b
	\end{array}\right]$
\end{center}

Nevertheless, we have that its determinant is given by:

\begin{center}
	$\left|A\right| = \left|\begin{array}{cc}
		\sqrt{3}b & 3b\\
		b & \sqrt{3}b
	\end{array}\right| = \sqrt{3}b\cdot\sqrt{3}b - 3b\cdot b = (\sqrt{3})^{2}b^{2} - 3b^{2} = 3b^{2} - 3b^{2} = 0$.
\end{center}

So, the previous matrix given by $A$ is a matrix that belongs to $G$ but such matrix does not have a multiplicative inverse because $\det(A) = 0$. Therefore, we conclude that $G$ is not a group. \hspace{0.1cm} $\square$

\begin{Prob}
	Let
	
	\begin{center}
		$G = \biggl\{\left[\begin{array}{cc}
		a & 3b\\
		b & a
		\end{array}\right] : a$ and $b$ are rational numbers and $a \neq 0$ and $b \neq 0\biggr\} $
	\end{center}
	
	Prove that $G$ is a group with respect to multiplication.
\end{Prob}

\textbf{Proof:} In order to show that $G$ represents a group, we need to show that $G$ (together with the matrix multiplication as the set operation) satisfies the following group conditions:

\begin{enumerate}
	\item \textbf{Closure Law}: If $A$ and $B$ are elements of $G$, then $AB$ is also an element of $G$. 
	
	\item \textbf{Associative Law}: If $A, B$ and $C$ are elements of $G$, then $(AB)C = A(BC)$.
	
	\item \textbf{Existence of unit}: There is an element $I$ in $G$ such that for all $A$ in $G$, we have that $IA = AI = A$.
	
	\item \textbf{Existence of inverse}: For every element $A$ in $G$, there is another element $B$ in $G$ such that $AB = BA = I$.
\end{enumerate}

Let us show that each of the previous conditions are satisfied by $G$.\\

\underline{Closure Law:} Let us take two matrices $A$ and $B$ in $G$ such that:

\begin{center}
	$A = \left(\begin{array}{cc}
	a & 3b\\
	b & a
	\end{array}\right)$ \hspace{0.1cm} $\wedge$ \hspace{0.1cm} $B = \left(\begin{array}{cc}
		c & 3d\\
		d & c
	\end{array}\right)$
\end{center}

Let us compute the multiplication of $A$ and $B$ in the following way:

\begin{center}
	$AB = \left(\begin{array}{cc}
	a & 3b\\
	b & a
	\end{array}\right) \left(\begin{array}{cc}
	c & 3d\\
	d & c
	\end{array}\right) = \left(\begin{array}{cc}
	ac + 3bd & 3ad+3bc \\
	bc+ad & 3bd+ac
	\end{array}\right) = \left(\begin{array}{cc}
	ac + 3bd & 3(ad+bc) \\
	ad+bc & ac+3bd
	\end{array}\right)$
\end{center}

Therefore, if we take the following substitution:

\begin{center}
	$e = ac + 3bd$ \hspace{0.1cm} and \hspace{0.1cm} $f = ad+bc$
\end{center}

Then, we have that the matrix $AB$ is equal to the following expression:

\begin{center}
	$AB = \left(\begin{array}{cc}
	ac + 3bd & 3(ad+bc) \\
	ad+bc & ac+3bd
	\end{array}\right) = \left(\begin{array}{cc}
	e & 3f \\
	f & e
	\end{array}\right)$
\end{center}

Thus, the matrix $AB$ is an element of $G$ and therefore $G$ is closed under matrix multiplication.\\

\underline{Associativity:} Matrix multiplication is an associative operation on the set of all square matrices $2\times2$ and therefore, as $G$ is a subset of such set, then we can conclude that the associative law is satisfied in $G$ because it is inherited from the set of all square matrices $2\times2$ under multiplication.\\

\underline{Existence of unit:} The existence of such matrix can be shown by exhibiting the identity matrix and showing that it belongs to $G$. Effectively, take into account the following matrix:

\begin{center}
	$I = \left(\begin{array}{cc}
		1 & 0\\
		0 & 1
	\end{array}\right) = \left(\begin{array}{cc}
	1 & 3(0)\\
	0 & 1
\end{array}\right) \in \hspace{0.1cm} G$
\end{center}

So, we have shown that the matrix $I$ has the form of any matrix in $G$. Therefore, $I$ belongs to $G$, which means that $G$ has a unit element.\\

\underline{Existence of inverse:} Now, following the same procedure as in the previous problem, we have that if $A$ is a matrix of $G$ such that:

\begin{center}
	$A = \left(\begin{array}{cc}
	a & 3b\\
	b & a
	\end{array}\right)$
\end{center}

Then, in order for $A$ to have an inverse, we need to assure that $\det(A) \neq 0$. Thus, we know that:

\begin{center}
	$\det(A) = \left|\begin{array}{cc}
	a & 3b\\
	b & a
	\end{array}\right| = a^{2} - 3b^{2}$
\end{center}

So, the only way in which $A$ has not a multiplicative inverse is that:

\begin{center}
	$a^{2}-3b^{2} \iff a^{2} = 3b^{2} \iff \pm a = \pm \sqrt{3}b$
\end{center}

Nevertheless, the previous relation is not possible because all entries of the matrix $A$ must be rational numbers and $a = \sqrt{3}b$ is not rational, because we have that both $a$ and $b$ are rational but $\sqrt{3}$ is not, so the product of $b$ and $\sqrt{3}$ is an irrational number which makes $a$ irrational and rational (contradiction). Thus, the determinant of any matrix in $G$ must be different from zero and therefore every matrix in $G$ has an multiplicative inverse in $G$.\\

For all the previous remarks, we conclude that effectively $G$ represents a group. \hspace{0.1cm} $\square$ 

\end{document}